%% Generated by Sphinx.
\def\sphinxdocclass{report}
\documentclass[letterpaper,10pt,english]{sphinxmanual}
\ifdefined\pdfpxdimen
   \let\sphinxpxdimen\pdfpxdimen\else\newdimen\sphinxpxdimen
\fi \sphinxpxdimen=.75bp\relax
\ifdefined\pdfimageresolution
    \pdfimageresolution= \numexpr \dimexpr1in\relax/\sphinxpxdimen\relax
\fi
%% let collapsible pdf bookmarks panel have high depth per default
\PassOptionsToPackage{bookmarksdepth=5}{hyperref}

\PassOptionsToPackage{warn}{textcomp}
\usepackage[utf8]{inputenc}
\ifdefined\DeclareUnicodeCharacter
% support both utf8 and utf8x syntaxes
  \ifdefined\DeclareUnicodeCharacterAsOptional
    \def\sphinxDUC#1{\DeclareUnicodeCharacter{"#1}}
  \else
    \let\sphinxDUC\DeclareUnicodeCharacter
  \fi
  \sphinxDUC{00A0}{\nobreakspace}
  \sphinxDUC{2500}{\sphinxunichar{2500}}
  \sphinxDUC{2502}{\sphinxunichar{2502}}
  \sphinxDUC{2514}{\sphinxunichar{2514}}
  \sphinxDUC{251C}{\sphinxunichar{251C}}
  \sphinxDUC{2572}{\textbackslash}
\fi
\usepackage{cmap}
\usepackage[T1]{fontenc}
\usepackage{amsmath,amssymb,amstext}
\usepackage{babel}



\usepackage{tgtermes}
\usepackage{tgheros}
\renewcommand{\ttdefault}{txtt}



\usepackage[Bjarne]{fncychap}
\usepackage{sphinx}

\fvset{fontsize=auto}
\usepackage{geometry}


% Include hyperref last.
\usepackage{hyperref}
% Fix anchor placement for figures with captions.
\usepackage{hypcap}% it must be loaded after hyperref.
% Set up styles of URL: it should be placed after hyperref.
\urlstyle{same}

\addto\captionsenglish{\renewcommand{\contentsname}{Contents:}}

\usepackage{sphinxmessages}
\setcounter{tocdepth}{1}



\title{Hypergraph Analysis Toolbox}
\date{Dec 01, 2022}
\release{0.0.1}
\author{Joshua Pickard}
\newcommand{\sphinxlogo}{\vbox{}}
\renewcommand{\releasename}{Release}
\makeindex
\begin{document}

\ifdefined\shorthandoff
  \ifnum\catcode`\=\string=\active\shorthandoff{=}\fi
  \ifnum\catcode`\"=\active\shorthandoff{"}\fi
\fi

\pagestyle{empty}
\sphinxmaketitle
\pagestyle{plain}
\sphinxtableofcontents
\pagestyle{normal}
\phantomsection\label{\detokenize{index::doc}}


\noindent{\hspace*{\fill}\sphinxincludegraphics{{index_dyadic_decomp}.png}\hspace*{\fill}}


\chapter{\sphinxstyleemphasis{Important Note}}
\label{\detokenize{index:important-note}}
\sphinxAtStartPar
The software for HAT is complete, but the online documentation is a work in progress. Currently, the software is only accessable via git and the Mathworks file exchange, but it will be published via PiPy shortly.


\chapter{Introduction}
\label{\detokenize{index:introduction}}
\sphinxAtStartPar
Hypergraph Analysis Toolbox (HAT) is a software suite for the analysis and visualization of hypergraphs and
higher order structures. Motivated to investigate Pore\sphinxhyphen{}C data, HAT is intended as a general prupose, versatile
software for hypergraph construction, visualization, and analysis. HAT addresses the following hypergraph
problems:
\begin{enumerate}
\sphinxsetlistlabels{\arabic}{enumi}{enumii}{}{.}%
\item {} 
\sphinxAtStartPar
Construction

\item {} 
\sphinxAtStartPar
Visualization

\item {} 
\sphinxAtStartPar
Expansion and numeric representation

\item {} 
\sphinxAtStartPar
Structral Properties

\item {} 
\sphinxAtStartPar
Controllability

\item {} 
\sphinxAtStartPar
Similarity Measures

\end{enumerate}

\sphinxAtStartPar
The capabilities and use cases of HAT are outlined in \sphinxhref{https://drive.google.com/file/d/1Mx8ifUtjR05ufhTXc5QgYKlwmfQjfItv/view?usp=share\_link}{this short notice}.

\sphinxstepscope


\section{Tutorials}
\label{\detokenize{tutorials:tutorials}}\label{\detokenize{tutorials::doc}}
\sphinxAtStartPar
This page contains a series of demonstrations on how to use HAT.


\subsection{MATLAB}
\label{\detokenize{tutorials:matlab}}\begin{enumerate}
\sphinxsetlistlabels{\arabic}{enumi}{enumii}{}{.}%
\item {} 
\sphinxAtStartPar
\sphinxhref{https://drive.matlab.com/sharing/7d77b042-c3cb-4ae9-8c06-515089fbccef}{Introduction}

\item {} 
\sphinxAtStartPar
\sphinxhref{https://drive.matlab.com/sharing/999692ae-26df-4b34-9cd4-c31af10d0bc3}{Multi\sphinxhyphen{}way Interactions}

\end{enumerate}


\subsection{Python}
\label{\detokenize{tutorials:python}}
\sphinxstepscope


\section{HAT Documentation}
\label{\detokenize{HAT:hat-documentation}}\label{\detokenize{HAT::doc}}

\subsection{Submodules}
\label{\detokenize{HAT:submodules}}

\subsection{HAT.Hypergraph module}
\label{\detokenize{HAT:module-HAT.Hypergraph}}\label{\detokenize{HAT:hat-hypergraph-module}}\index{module@\spxentry{module}!HAT.Hypergraph@\spxentry{HAT.Hypergraph}}\index{HAT.Hypergraph@\spxentry{HAT.Hypergraph}!module@\spxentry{module}}\index{Hypergraph (class in HAT.Hypergraph)@\spxentry{Hypergraph}\spxextra{class in HAT.Hypergraph}}

\begin{fulllineitems}
\phantomsection\label{\detokenize{HAT:HAT.Hypergraph.Hypergraph}}
\pysigstartsignatures
\pysiglinewithargsret{\sphinxbfcode{\sphinxupquote{class\DUrole{w}{  }}}\sphinxcode{\sphinxupquote{HAT.Hypergraph.}}\sphinxbfcode{\sphinxupquote{Hypergraph}}}{\emph{\DUrole{n}{im}}, \emph{\DUrole{n}{ew}\DUrole{o}{=}\DUrole{default_value}{None}}, \emph{\DUrole{n}{nw}\DUrole{o}{=}\DUrole{default_value}{None}}}{}
\pysigstopsignatures
\sphinxAtStartPar
Bases: \sphinxcode{\sphinxupquote{object}}

\sphinxAtStartPar
This is the base class representing a Hypergraph object. It is the primary entry point and
provides an interface to functions implemented in HAT’s other modules. The underlying data
structure of this class is an incidence matrix, but many methods exploit tensor representation
of uniform hypergraphs.

\sphinxAtStartPar
Formally, a Hypergraph \(H=(V,E)\) is a set of vertices \(V\) and a set of edges \(E\)
where each edge \(e\in E\) is defined \(e\subseteq V.\) In contrast to a graph, a hypergraph
edge \(e\) can contain any number of vertices, which allows for efficient representation of multi\sphinxhyphen{}way
relationships.

\sphinxAtStartPar
In a uniform Hypergraph, all edges contain the same number of vertices. Uniform hypergraphs are represnted
as tensors, which precisely model multi\sphinxhyphen{}way interactions.
\begin{quote}\begin{description}
\sphinxlineitem{Parameters}\begin{itemize}
\item {} 
\sphinxAtStartPar
\sphinxstyleliteralstrong{\sphinxupquote{im}} \textendash{} Incidence matrix

\item {} 
\sphinxAtStartPar
\sphinxstyleliteralstrong{\sphinxupquote{ew}} \textendash{} Edge weight vector

\item {} 
\sphinxAtStartPar
\sphinxstyleliteralstrong{\sphinxupquote{nw}} \textendash{} Node weight vector

\end{itemize}

\end{description}\end{quote}
\index{draw() (HAT.Hypergraph.Hypergraph method)@\spxentry{draw()}\spxextra{HAT.Hypergraph.Hypergraph method}}

\begin{fulllineitems}
\phantomsection\label{\detokenize{HAT:HAT.Hypergraph.Hypergraph.draw}}
\pysigstartsignatures
\pysiglinewithargsret{\sphinxbfcode{\sphinxupquote{draw}}}{\emph{\DUrole{n}{shadeRows}\DUrole{o}{=}\DUrole{default_value}{True}}, \emph{\DUrole{n}{connectNodes}\DUrole{o}{=}\DUrole{default_value}{True}}, \emph{\DUrole{n}{dpi}\DUrole{o}{=}\DUrole{default_value}{200}}, \emph{\DUrole{n}{edgeColors}\DUrole{o}{=}\DUrole{default_value}{None}}}{}
\pysigstopsignatures
\sphinxAtStartPar
This function draws the incidence matrix of the hypergraph object. It calls the function
\sphinxcode{\sphinxupquote{HAT.draw.incidencePlot}}, but is provided to generate the plot directly from the object.
\begin{quote}\begin{description}
\sphinxlineitem{Parameters}\begin{itemize}
\item {} 
\sphinxAtStartPar
\sphinxstyleliteralstrong{\sphinxupquote{shadeRows}} \textendash{} shade rows (bool)

\item {} 
\sphinxAtStartPar
\sphinxstyleliteralstrong{\sphinxupquote{connectNodes}} \textendash{} connect nodes in each hyperedge (bool)

\item {} 
\sphinxAtStartPar
\sphinxstyleliteralstrong{\sphinxupquote{dpi}} \textendash{} the resolution of the image (int)

\item {} 
\sphinxAtStartPar
\sphinxstyleliteralstrong{\sphinxupquote{edgeColors}} \textendash{} The colors of edges represented in the incidence matrix. This is random by default

\end{itemize}

\sphinxlineitem{Returns}
\sphinxAtStartPar
\sphinxcode{\sphinxupquote{matplotlib}} axes with figure drawn on to it

\end{description}\end{quote}

\end{fulllineitems}

\index{dual() (HAT.Hypergraph.Hypergraph method)@\spxentry{dual()}\spxextra{HAT.Hypergraph.Hypergraph method}}

\begin{fulllineitems}
\phantomsection\label{\detokenize{HAT:HAT.Hypergraph.Hypergraph.dual}}
\pysigstartsignatures
\pysiglinewithargsret{\sphinxbfcode{\sphinxupquote{dual}}}{}{}
\pysigstopsignatures
\sphinxAtStartPar
The dual hypergraph is constructed.
\begin{quote}\begin{description}
\sphinxlineitem{Returns}
\sphinxAtStartPar
Hypergraph object

\sphinxlineitem{Return type}
\sphinxAtStartPar
\sphinxstyleemphasis{Hypergraph}

\end{description}\end{quote}

\sphinxAtStartPar
Let \(H=(V,E)\) be a hypergraph. In the dual hypergraph each original edge \(e\in E\)
is represented as a vertex and each original vertex \(v\in E\) is represented as an edge. Numerically, the
transpose of the incidence matrix of a hypergraph is the incidence matrix of the dual hypergraph.
\subsubsection*{References}

\end{fulllineitems}

\index{cliqueGraph() (HAT.Hypergraph.Hypergraph method)@\spxentry{cliqueGraph()}\spxextra{HAT.Hypergraph.Hypergraph method}}

\begin{fulllineitems}
\phantomsection\label{\detokenize{HAT:HAT.Hypergraph.Hypergraph.cliqueGraph}}
\pysigstartsignatures
\pysiglinewithargsret{\sphinxbfcode{\sphinxupquote{cliqueGraph}}}{}{}
\pysigstopsignatures
\sphinxAtStartPar
The clique expansion graph is constructed.
\begin{quote}\begin{description}
\sphinxlineitem{Returns}
\sphinxAtStartPar
Clique expanded graph

\sphinxlineitem{Return type}
\sphinxAtStartPar
\sphinxstyleemphasis{networkx.graph}

\end{description}\end{quote}

\sphinxAtStartPar
The clique expansion algorithm constructs a \sphinxstyleemphasis{graph} on the same set of vertices as the hypergraph by defining an
edge set where every pair of vertices contained within the same edge in the hypergraph have an edge between them
in the graph. Given a hypergraph \(H=(V,E_h)\), then the corresponding clique graph is \(C=(V,E_c)\) where
\(E_c\) is defined
\begin{equation*}
\begin{split}E_c = \{(v_i, v_j) |\ \exists\  e\in E_h \text{ where } v_i, v_j\in e\}.\end{split}
\end{equation*}
\sphinxAtStartPar
This is called clique expansion because the vertices contained in each \(h\in E_h\) forms a clique in \(C\).
While the map from \(H\) to \(C\) is well\sphinxhyphen{}defined, the transformation to a clique graph is a lossy process,
so the hypergraph structure of \(H\) cannot be uniquely recovered from the clique graph \(C\) alone {[}1{]}.
\subsubsection*{References}

\end{fulllineitems}

\index{lineGraph() (HAT.Hypergraph.Hypergraph method)@\spxentry{lineGraph()}\spxextra{HAT.Hypergraph.Hypergraph method}}

\begin{fulllineitems}
\phantomsection\label{\detokenize{HAT:HAT.Hypergraph.Hypergraph.lineGraph}}
\pysigstartsignatures
\pysiglinewithargsret{\sphinxbfcode{\sphinxupquote{lineGraph}}}{}{}
\pysigstopsignatures
\sphinxAtStartPar
The line graph, which is the clique expansion of the dual graph, is constructed.
\begin{quote}\begin{description}
\sphinxlineitem{Returns}
\sphinxAtStartPar
Line graph

\sphinxlineitem{Return type}
\sphinxAtStartPar
\sphinxstyleemphasis{networkx.graph}

\end{description}\end{quote}
\subsubsection*{References}

\end{fulllineitems}

\index{starGraph() (HAT.Hypergraph.Hypergraph method)@\spxentry{starGraph()}\spxextra{HAT.Hypergraph.Hypergraph method}}

\begin{fulllineitems}
\phantomsection\label{\detokenize{HAT:HAT.Hypergraph.Hypergraph.starGraph}}
\pysigstartsignatures
\pysiglinewithargsret{\sphinxbfcode{\sphinxupquote{starGraph}}}{}{}
\pysigstopsignatures
\sphinxAtStartPar
The star graph representation is constructed.
\begin{quote}\begin{description}
\sphinxlineitem{Returns}
\sphinxAtStartPar
Star graph

\sphinxlineitem{Return type}
\sphinxAtStartPar
\sphinxstyleemphasis{networkx.graph}

\end{description}\end{quote}

\sphinxAtStartPar
The star expansion of \({H}=({V},{E}_h)\) constructs a bipartite graph \({S}=\{{V}_s,{E}_s\}\)
by introducing a new set of vertices \({V}_s={V}\cup {E}_h\) where some vertices in the star graph
represent hyperedges of the original hypergraph. There exists an edge between each vertex \(v,e\in {V}_s\)
when \(v\in {V}\), \(e\in {E}_h,\) and \(v\in e\). Each hyperedge in \({E}_h\) induces
a star in \(S\). This is a lossless process, so the hypergraph structure of \(H\) is well\sphinxhyphen{}defined{]}
given a star graph \(S\).
\subsubsection*{References}

\end{fulllineitems}

\index{laplacianMatrix() (HAT.Hypergraph.Hypergraph method)@\spxentry{laplacianMatrix()}\spxextra{HAT.Hypergraph.Hypergraph method}}

\begin{fulllineitems}
\phantomsection\label{\detokenize{HAT:HAT.Hypergraph.Hypergraph.laplacianMatrix}}
\pysigstartsignatures
\pysiglinewithargsret{\sphinxbfcode{\sphinxupquote{laplacianMatrix}}}{\emph{\DUrole{n}{type}\DUrole{o}{=}\DUrole{default_value}{\textquotesingle{}Bolla\textquotesingle{}}}}{}
\pysigstopsignatures
\sphinxAtStartPar
This function returns a version of the higher order Laplacian matrix of the hypergraph.
\begin{quote}\begin{description}
\sphinxlineitem{Parameters}
\sphinxAtStartPar
\sphinxstyleliteralstrong{\sphinxupquote{type}} (\sphinxstyleliteralemphasis{\sphinxupquote{str}}\sphinxstyleliteralemphasis{\sphinxupquote{, }}\sphinxstyleliteralemphasis{\sphinxupquote{optional}}) \textendash{} Indicates which version of the Laplacin matrix to return. It defaults to \sphinxcode{\sphinxupquote{Bolla}} {[}1{]}, but \sphinxcode{\sphinxupquote{Rodriguez}} {[}2,3{]} and \sphinxcode{\sphinxupquote{Zhou}} {[}4{]} are valid arguments as well.

\sphinxlineitem{Returns}
\sphinxAtStartPar
Laplacian matrix

\sphinxlineitem{Return type}
\sphinxAtStartPar
\sphinxstyleemphasis{ndarray}

\end{description}\end{quote}

\sphinxAtStartPar
Several version of the hypergraph Laplacian are defined in {[}1\sphinxhyphen{}4{]}. These aim to capture
the higher order structure as a matrix. This function serves as a wrapper to call functions
that generate different specific Laplacians (See \sphinxcode{\sphinxupquote{bollaLaplacian()}}, \sphinxcode{\sphinxupquote{rodriguezLaplacian()}},
and \sphinxcode{\sphinxupquote{zhouLaplacian()}}).
\subsubsection*{References}

\end{fulllineitems}

\index{bollaLaplacian() (HAT.Hypergraph.Hypergraph method)@\spxentry{bollaLaplacian()}\spxextra{HAT.Hypergraph.Hypergraph method}}

\begin{fulllineitems}
\phantomsection\label{\detokenize{HAT:HAT.Hypergraph.Hypergraph.bollaLaplacian}}
\pysigstartsignatures
\pysiglinewithargsret{\sphinxbfcode{\sphinxupquote{bollaLaplacian}}}{}{}
\pysigstopsignatures
\sphinxAtStartPar
This function constructs the hypergraph Laplacian according to {[}1{]}.
\begin{quote}\begin{description}
\sphinxlineitem{Returns}
\sphinxAtStartPar
Bolla Laplacian matrix

\sphinxlineitem{Return type}
\sphinxAtStartPar
\sphinxstyleemphasis{ndarray}

\end{description}\end{quote}
\subsubsection*{References}

\end{fulllineitems}

\index{rodriguezLaplacian() (HAT.Hypergraph.Hypergraph method)@\spxentry{rodriguezLaplacian()}\spxextra{HAT.Hypergraph.Hypergraph method}}

\begin{fulllineitems}
\phantomsection\label{\detokenize{HAT:HAT.Hypergraph.Hypergraph.rodriguezLaplacian}}
\pysigstartsignatures
\pysiglinewithargsret{\sphinxbfcode{\sphinxupquote{rodriguezLaplacian}}}{}{}
\pysigstopsignatures
\sphinxAtStartPar
This function constructs the hypergraph Laplacian according to {[}1, 2{]}.
\begin{quote}\begin{description}
\sphinxlineitem{Returns}
\sphinxAtStartPar
Rodriguez Laplacian matrix

\sphinxlineitem{Return type}
\sphinxAtStartPar
\sphinxstyleemphasis{ndarray}

\end{description}\end{quote}
\subsubsection*{References}

\end{fulllineitems}

\index{zhouLaplacian() (HAT.Hypergraph.Hypergraph method)@\spxentry{zhouLaplacian()}\spxextra{HAT.Hypergraph.Hypergraph method}}

\begin{fulllineitems}
\phantomsection\label{\detokenize{HAT:HAT.Hypergraph.Hypergraph.zhouLaplacian}}
\pysigstartsignatures
\pysiglinewithargsret{\sphinxbfcode{\sphinxupquote{zhouLaplacian}}}{}{}
\pysigstopsignatures
\sphinxAtStartPar
This function constructs the hypergraph Laplacian according to {[}1{]}.
\begin{quote}\begin{description}
\sphinxlineitem{Returns}
\sphinxAtStartPar
Zhou Laplacian matrix

\sphinxlineitem{Return type}
\sphinxAtStartPar
\sphinxstyleemphasis{ndarray}

\end{description}\end{quote}
\subsubsection*{References}

\end{fulllineitems}

\index{adjTensor() (HAT.Hypergraph.Hypergraph method)@\spxentry{adjTensor()}\spxextra{HAT.Hypergraph.Hypergraph method}}

\begin{fulllineitems}
\phantomsection\label{\detokenize{HAT:HAT.Hypergraph.Hypergraph.adjTensor}}
\pysigstartsignatures
\pysiglinewithargsret{\sphinxbfcode{\sphinxupquote{adjTensor}}}{}{}
\pysigstopsignatures
\sphinxAtStartPar
This constructs the adjacency tensor for uniform hypergraphs.
\begin{quote}\begin{description}
\sphinxlineitem{Returns}
\sphinxAtStartPar
Adjacency Tensor

\sphinxlineitem{Return type}
\sphinxAtStartPar
\sphinxstyleemphasis{ndarray}

\end{description}\end{quote}

\sphinxAtStartPar
The adjacency tensor \(A\) of a \(k-\) is the multi\sphinxhyphen{}way, hypergraph analog of the pairwise, graph
adjacency matrix. It is defined as a \(k-\) mode tensor ( \(k-\) dimensional matrix):
\begin{equation*}
\begin{split}A \in \mathbf{R}^{ \overbrace{n \times \dots \times n}^{k \text{ times}}} \text{ where }{A}_{j_1\dots j_k} = \begin{cases} \frac{1}{(k-1)!} & \text{if }(j_1,\dots,j_k)\in {E}_h \\ 0 & \text{otherwise} \end{cases},\end{split}
\end{equation*}
\sphinxAtStartPar
as found in equation 8 of {[}1{]}.
\subsubsection*{References}

\end{fulllineitems}

\index{degreeTensor() (HAT.Hypergraph.Hypergraph method)@\spxentry{degreeTensor()}\spxextra{HAT.Hypergraph.Hypergraph method}}

\begin{fulllineitems}
\phantomsection\label{\detokenize{HAT:HAT.Hypergraph.Hypergraph.degreeTensor}}
\pysigstartsignatures
\pysiglinewithargsret{\sphinxbfcode{\sphinxupquote{degreeTensor}}}{}{}
\pysigstopsignatures
\sphinxAtStartPar
This constructs the degree tensor for uniform hypergraphs.
\begin{quote}\begin{description}
\sphinxlineitem{Returns}
\sphinxAtStartPar
Degree Tensor

\sphinxlineitem{Return type}
\sphinxAtStartPar
\sphinxstyleemphasis{ndarray}

\end{description}\end{quote}
\begin{description}
\sphinxlineitem{The degree tensor \(D\) is the hypergraph analog of the degree matrix. For a \(k-\) order hypergraph}
\sphinxAtStartPar
\(H=(V,E)\) the degree tensor \(D\) is a diagonal supersymmetric tensor defined

\end{description}
\begin{equation*}
\begin{split}D \in \mathbf{R}^{ \overbrace{n \times \dots \times n}^{k \text{ times}}} \text{ where }{D}_{i\dots i} = degree(v_i) \text{ for all } v_i\in V\end{split}
\end{equation*}\subsubsection*{References}

\end{fulllineitems}

\index{laplacianTensor() (HAT.Hypergraph.Hypergraph method)@\spxentry{laplacianTensor()}\spxextra{HAT.Hypergraph.Hypergraph method}}

\begin{fulllineitems}
\phantomsection\label{\detokenize{HAT:HAT.Hypergraph.Hypergraph.laplacianTensor}}
\pysigstartsignatures
\pysiglinewithargsret{\sphinxbfcode{\sphinxupquote{laplacianTensor}}}{}{}
\pysigstopsignatures
\sphinxAtStartPar
This constructs the Laplacian tensor for uniform hypergraphs.
\begin{quote}\begin{description}
\sphinxlineitem{Returns}
\sphinxAtStartPar
Laplcian Tensor

\sphinxlineitem{Return type}
\sphinxAtStartPar
\sphinxstyleemphasis{ndarray}

\end{description}\end{quote}

\sphinxAtStartPar
The Laplacian tensor is the tensor analog of the Laplacian matrix for graphs, and it is
defined equivalently. For a hypergraph \(H=(V,E)\) with an adjacency tensor \(A\)
and degree tensor \(D\), the Laplacian tensor is
\begin{equation*}
\begin{split}L = D - A\end{split}
\end{equation*}\subsubsection*{References}

\end{fulllineitems}

\index{tensorEntropy() (HAT.Hypergraph.Hypergraph method)@\spxentry{tensorEntropy()}\spxextra{HAT.Hypergraph.Hypergraph method}}

\begin{fulllineitems}
\phantomsection\label{\detokenize{HAT:HAT.Hypergraph.Hypergraph.tensorEntropy}}
\pysigstartsignatures
\pysiglinewithargsret{\sphinxbfcode{\sphinxupquote{tensorEntropy}}}{}{}
\pysigstopsignatures
\sphinxAtStartPar
Computes hypergraph entropy based on the singular values of the Laplacian tensor.
\begin{quote}\begin{description}
\sphinxlineitem{Returns}
\sphinxAtStartPar
tensor entropy

\sphinxlineitem{Return type}
\sphinxAtStartPar
\sphinxstyleemphasis{float}

\end{description}\end{quote}

\sphinxAtStartPar
Uniform hypergraph entropy is defined as the entropy of the higher order singular
values of the Laplacian matrix {[}1{]}.
\subsubsection*{References}

\end{fulllineitems}

\index{matrixEntropy() (HAT.Hypergraph.Hypergraph method)@\spxentry{matrixEntropy()}\spxextra{HAT.Hypergraph.Hypergraph method}}

\begin{fulllineitems}
\phantomsection\label{\detokenize{HAT:HAT.Hypergraph.Hypergraph.matrixEntropy}}
\pysigstartsignatures
\pysiglinewithargsret{\sphinxbfcode{\sphinxupquote{matrixEntropy}}}{\emph{\DUrole{n}{type}\DUrole{o}{=}\DUrole{default_value}{\textquotesingle{}Rodriguez\textquotesingle{}}}}{}
\pysigstopsignatures
\sphinxAtStartPar
Computes hypergraph entropy based on the eigenvalues values of the Laplacian matrix.
\begin{quote}\begin{description}
\sphinxlineitem{Parameters}
\sphinxAtStartPar
\sphinxstyleliteralstrong{\sphinxupquote{type}} (\sphinxstyleliteralemphasis{\sphinxupquote{str}}\sphinxstyleliteralemphasis{\sphinxupquote{, }}\sphinxstyleliteralemphasis{\sphinxupquote{optional}}) \textendash{} Type of hypergraph Laplacian matrix. This defaults to ‘Rodriguez’ and other
choices inclue ‘Bolla’ and ‘Zhou’ (See: \sphinxcode{\sphinxupquote{laplacianMatrix()}}).

\sphinxlineitem{Returns}
\sphinxAtStartPar
Matrix based hypergraph entropy

\sphinxlineitem{Return type}
\sphinxAtStartPar
\sphinxstyleemphasis{float}

\end{description}\end{quote}

\sphinxAtStartPar
Matrix entropy of a hypergraph is defined as the entropy of the eigenvalues of the
hypergraph Laplacian matrix {[}1{]}. This may be applied to any version of the Laplacian matrix.
\subsubsection*{References}

\end{fulllineitems}

\index{avgDistance() (HAT.Hypergraph.Hypergraph method)@\spxentry{avgDistance()}\spxextra{HAT.Hypergraph.Hypergraph method}}

\begin{fulllineitems}
\phantomsection\label{\detokenize{HAT:HAT.Hypergraph.Hypergraph.avgDistance}}
\pysigstartsignatures
\pysiglinewithargsret{\sphinxbfcode{\sphinxupquote{avgDistance}}}{}{}
\pysigstopsignatures
\sphinxAtStartPar
Computes the average pairwise distance between any 2 vertices in the hypergraph.
\begin{quote}\begin{description}
\sphinxlineitem{Returns}
\sphinxAtStartPar
avgDist

\sphinxlineitem{Return type}
\sphinxAtStartPar
float

\end{description}\end{quote}

\sphinxAtStartPar
The hypergraph is clique expanded to a graph object, and the average shortest path on
the clique expanded graph is returned.

\end{fulllineitems}

\index{ctrbk() (HAT.Hypergraph.Hypergraph method)@\spxentry{ctrbk()}\spxextra{HAT.Hypergraph.Hypergraph method}}

\begin{fulllineitems}
\phantomsection\label{\detokenize{HAT:HAT.Hypergraph.Hypergraph.ctrbk}}
\pysigstartsignatures
\pysiglinewithargsret{\sphinxbfcode{\sphinxupquote{ctrbk}}}{\emph{\DUrole{n}{inputVxc}}}{}
\pysigstopsignatures
\sphinxAtStartPar
Compute the reduced controllability matrix for \(k-\) uniform hypergraphs.
\begin{quote}\begin{description}
\sphinxlineitem{Parameters}
\sphinxAtStartPar
\sphinxstyleliteralstrong{\sphinxupquote{inputVxc}} (\sphinxstyleemphasis{ndarray}) \textendash{} List of vertices that may be controlled

\sphinxlineitem{Returns}
\sphinxAtStartPar
Controllability matrix

\sphinxlineitem{Return type}
\sphinxAtStartPar
\sphinxstyleemphasis{ndarray}

\end{description}\end{quote}
\subsubsection*{References}

\end{fulllineitems}

\index{bMatrix() (HAT.Hypergraph.Hypergraph method)@\spxentry{bMatrix()}\spxextra{HAT.Hypergraph.Hypergraph method}}

\begin{fulllineitems}
\phantomsection\label{\detokenize{HAT:HAT.Hypergraph.Hypergraph.bMatrix}}
\pysigstartsignatures
\pysiglinewithargsret{\sphinxbfcode{\sphinxupquote{bMatrix}}}{\emph{\DUrole{n}{inputVxc}}}{}
\pysigstopsignatures
\sphinxAtStartPar
Constructs controllability \(B\) matrix commonly used in the linear control system
\begin{equation*}
\begin{split}\frac{dx}{dt} = Ax+Bu\end{split}
\end{equation*}\begin{quote}\begin{description}
\sphinxlineitem{Parameters}
\sphinxAtStartPar
\sphinxstyleliteralstrong{\sphinxupquote{inputVxc}} (\sphinxstyleemphasis{ndarray}) \textendash{} a list of input control nodes

\sphinxlineitem{Returns}
\sphinxAtStartPar
control matrix

\sphinxlineitem{Return type}
\sphinxAtStartPar
\sphinxstyleemphasis{ndarray}

\end{description}\end{quote}
\subsubsection*{References}

\end{fulllineitems}

\index{clusteringCoef() (HAT.Hypergraph.Hypergraph method)@\spxentry{clusteringCoef()}\spxextra{HAT.Hypergraph.Hypergraph method}}

\begin{fulllineitems}
\phantomsection\label{\detokenize{HAT:HAT.Hypergraph.Hypergraph.clusteringCoef}}
\pysigstartsignatures
\pysiglinewithargsret{\sphinxbfcode{\sphinxupquote{clusteringCoef}}}{}{}
\pysigstopsignatures
\sphinxAtStartPar
Computes clustering average clustering coefficient of the hypergraph.
\begin{quote}\begin{description}
\sphinxlineitem{Returns}
\sphinxAtStartPar
average clustering coefficient

\sphinxlineitem{Return type}
\sphinxAtStartPar
\sphinxstyleemphasis{float}

\end{description}\end{quote}

\sphinxAtStartPar
For a uniform hypergraph, the clustering coefficient of a vertex \(v_i\)
is defined as the number of edges the vertex participates in (i.e. \(deg(v_i)\)) divided
by the number of \(k-\) and its neighbors
(See equation 31 in {[}1{]}). This is written
\begin{equation*}
\begin{split}C_i = \frac{deg(v_i)}{\binom{|N_i|}{k}}\end{split}
\end{equation*}
\sphinxAtStartPar
where \(N_i\) is the set of neighbors or vertices adjacent to \(v_i\). The hypergraph
clustering coefficient computed here is the average clustering coefficient for all vertices,
written
\begin{equation*}
\begin{split}C=\sum_{i=1}^nC_i\end{split}
\end{equation*}\subsubsection*{References}

\end{fulllineitems}

\index{centrality() (HAT.Hypergraph.Hypergraph method)@\spxentry{centrality()}\spxextra{HAT.Hypergraph.Hypergraph method}}

\begin{fulllineitems}
\phantomsection\label{\detokenize{HAT:HAT.Hypergraph.Hypergraph.centrality}}
\pysigstartsignatures
\pysiglinewithargsret{\sphinxbfcode{\sphinxupquote{centrality}}}{\emph{\DUrole{n}{tol}\DUrole{o}{=}\DUrole{default_value}{0.0001}}, \emph{\DUrole{n}{maxIter}\DUrole{o}{=}\DUrole{default_value}{3000}}, \emph{\DUrole{n}{model}\DUrole{o}{=}\DUrole{default_value}{\textquotesingle{}LogExp\textquotesingle{}}}, \emph{\DUrole{n}{alpha}\DUrole{o}{=}\DUrole{default_value}{10}}}{}
\pysigstopsignatures
\sphinxAtStartPar
Computes node and edge centralities.
\begin{quote}\begin{description}
\sphinxlineitem{Parameters}\begin{itemize}
\item {} 
\sphinxAtStartPar
\sphinxstyleliteralstrong{\sphinxupquote{tol}} (\sphinxstyleliteralemphasis{\sphinxupquote{\_type\_}}\sphinxstyleliteralemphasis{\sphinxupquote{, }}\sphinxstyleliteralemphasis{\sphinxupquote{optional}}) \textendash{} threshold tolerance for the convergence of the centrality measures, defaults to 1e\sphinxhyphen{}4

\item {} 
\sphinxAtStartPar
\sphinxstyleliteralstrong{\sphinxupquote{maxIter}} (\sphinxstyleliteralemphasis{\sphinxupquote{int}}\sphinxstyleliteralemphasis{\sphinxupquote{, }}\sphinxstyleliteralemphasis{\sphinxupquote{optional}}) \textendash{} maximum number of iterations for the centrality measures to converge in, defaults to 3000

\item {} 
\sphinxAtStartPar
\sphinxstyleliteralstrong{\sphinxupquote{model}} (\sphinxstyleliteralemphasis{\sphinxupquote{str}}\sphinxstyleliteralemphasis{\sphinxupquote{, }}\sphinxstyleliteralemphasis{\sphinxupquote{optional}}) \textendash{} the set of functions used to compute centrality. This defaults to ‘LogExp’, and other choices include
‘Linear’, ‘Max’ or a list of 4 custom function handles (See {[}1{]}).

\item {} 
\sphinxAtStartPar
\sphinxstyleliteralstrong{\sphinxupquote{alpha}} (\sphinxstyleliteralemphasis{\sphinxupquote{int}}\sphinxstyleliteralemphasis{\sphinxupquote{, }}\sphinxstyleliteralemphasis{\sphinxupquote{optional}}) \textendash{} Hyperparameter used for computing centrality (See {[}1{]}), defaults to 10

\end{itemize}

\sphinxlineitem{Returns}
\sphinxAtStartPar
vxcCentrality

\sphinxlineitem{Return type}
\sphinxAtStartPar
\sphinxstyleemphasis{ndarray} containing centrality scores for each vertex in the hypergraph

\sphinxlineitem{Returns}
\sphinxAtStartPar
edgeCentrality

\sphinxlineitem{Return type}
\sphinxAtStartPar
\sphinxstyleemphasis{ndarray} containing centrality scores for each edge in the hypergraph

\end{description}\end{quote}
\subsubsection*{References}

\end{fulllineitems}


\end{fulllineitems}



\subsection{HAT.HAT module}
\label{\detokenize{HAT:module-HAT.HAT}}\label{\detokenize{HAT:hat-hat-module}}\index{module@\spxentry{module}!HAT.HAT@\spxentry{HAT.HAT}}\index{HAT.HAT@\spxentry{HAT.HAT}!module@\spxentry{module}}\index{directSimilarity() (in module HAT.HAT)@\spxentry{directSimilarity()}\spxextra{in module HAT.HAT}}

\begin{fulllineitems}
\phantomsection\label{\detokenize{HAT:HAT.HAT.directSimilarity}}
\pysigstartsignatures
\pysiglinewithargsret{\sphinxcode{\sphinxupquote{HAT.HAT.}}\sphinxbfcode{\sphinxupquote{directSimilarity}}}{\emph{\DUrole{n}{HG1}}, \emph{\DUrole{n}{HG2}}, \emph{\DUrole{n}{measure}\DUrole{o}{=}\DUrole{default_value}{\textquotesingle{}Hamming\textquotesingle{}}}}{}
\pysigstopsignatures
\sphinxAtStartPar
This function computes the direct similarity between two uniform hypergraphs.
\begin{quote}\begin{description}
\sphinxlineitem{Parameters}\begin{itemize}
\item {} 
\sphinxAtStartPar
\sphinxstyleliteralstrong{\sphinxupquote{HG1}} (\sphinxstyleemphasis{Hypergraph}) \textendash{} Hypergraph 1

\item {} 
\sphinxAtStartPar
\sphinxstyleliteralstrong{\sphinxupquote{HG2}} (\sphinxstyleemphasis{Hypergraph}) \textendash{} Hypergraph 2

\item {} 
\sphinxAtStartPar
\sphinxstyleliteralstrong{\sphinxupquote{measure}} (\sphinxstyleliteralemphasis{\sphinxupquote{str}}\sphinxstyleliteralemphasis{\sphinxupquote{, }}\sphinxstyleliteralemphasis{\sphinxupquote{optional}}) \textendash{} This sepcifies which similarity measure to apply. It defaults to
\sphinxcode{\sphinxupquote{Hamming}}, and \sphinxcode{\sphinxupquote{Spectral\sphinxhyphen{}S}} and \sphinxcode{\sphinxupquote{Centrality}} are available as other options
as well.

\end{itemize}

\sphinxlineitem{Returns}
\sphinxAtStartPar
Hypergraph similarity

\sphinxlineitem{Return type}
\sphinxAtStartPar
\sphinxstyleemphasis{float}

\end{description}\end{quote}
\subsubsection*{References}

\end{fulllineitems}

\index{indirectSimilarity() (in module HAT.HAT)@\spxentry{indirectSimilarity()}\spxextra{in module HAT.HAT}}

\begin{fulllineitems}
\phantomsection\label{\detokenize{HAT:HAT.HAT.indirectSimilarity}}
\pysigstartsignatures
\pysiglinewithargsret{\sphinxcode{\sphinxupquote{HAT.HAT.}}\sphinxbfcode{\sphinxupquote{indirectSimilarity}}}{\emph{\DUrole{n}{G1}}, \emph{\DUrole{n}{G2}}, \emph{\DUrole{n}{measure}\DUrole{o}{=}\DUrole{default_value}{\textquotesingle{}Hamming\textquotesingle{}}}, \emph{\DUrole{n}{eps}\DUrole{o}{=}\DUrole{default_value}{0.01}}}{}
\pysigstopsignatures
\sphinxAtStartPar
This function computes the indirect similarity between two hypergraphs.
\begin{quote}\begin{description}
\sphinxlineitem{Parameters}\begin{itemize}
\item {} 
\sphinxAtStartPar
\sphinxstyleliteralstrong{\sphinxupquote{G1}} (\sphinxstyleemphasis{nx.Graph} or \sphinxstyleemphasis{ndarray}) \textendash{} Hypergraph 1 expansion

\item {} 
\sphinxAtStartPar
\sphinxstyleliteralstrong{\sphinxupquote{G2}} (\sphinxstyleemphasis{nx.Graph} or \sphinxstyleemphasis{ndarray}) \textendash{} Hypergraph 2 expansion

\item {} 
\sphinxAtStartPar
\sphinxstyleliteralstrong{\sphinxupquote{measure}} (\sphinxstyleemphasis{str}, optional) \textendash{} This specifies which similarity measure to apply. It defaults to \sphinxcode{\sphinxupquote{Hamming}} , and
\sphinxcode{\sphinxupquote{Jaccard}} , \sphinxcode{\sphinxupquote{deltaCon}} , \sphinxcode{\sphinxupquote{Spectral}} , and \sphinxcode{\sphinxupquote{Centrality}} are provided as well. When \sphinxcode{\sphinxupquote{Centrality}}
is used as the similarity measure, \sphinxcode{\sphinxupquote{G1}} and \sphinxcode{\sphinxupquote{G2}} should \sphinxstyleemphasis{ndarray} s of centrality values; Otherwise
\sphinxcode{\sphinxupquote{G1}} and \sphinxcode{\sphinxupquote{G2}} are \sphinxstyleemphasis{nx.Graph*s or *ndarray*} s as adjacency matrices.

\item {} 
\sphinxAtStartPar
\sphinxstyleliteralstrong{\sphinxupquote{eps}} (\sphinxstyleemphasis{float}, optional) \textendash{} a hyperparameter required for deltaCon similarity, defaults to 10e\sphinxhyphen{}3

\end{itemize}

\sphinxlineitem{Returns}
\sphinxAtStartPar
similarity measure

\sphinxlineitem{Return type}
\sphinxAtStartPar
\sphinxstyleemphasis{float}

\end{description}\end{quote}
\subsubsection*{References}

\end{fulllineitems}

\index{multicorrelations() (in module HAT.HAT)@\spxentry{multicorrelations()}\spxextra{in module HAT.HAT}}

\begin{fulllineitems}
\phantomsection\label{\detokenize{HAT:HAT.HAT.multicorrelations}}
\pysigstartsignatures
\pysiglinewithargsret{\sphinxcode{\sphinxupquote{HAT.HAT.}}\sphinxbfcode{\sphinxupquote{multicorrelations}}}{\emph{\DUrole{n}{D}}, \emph{\DUrole{n}{order}}, \emph{\DUrole{n}{mtype}\DUrole{o}{=}\DUrole{default_value}{\textquotesingle{}Drezner\textquotesingle{}}}, \emph{\DUrole{n}{idxs}\DUrole{o}{=}\DUrole{default_value}{None}}}{}
\pysigstopsignatures
\sphinxAtStartPar
This function computes the multicorrelation among pairwise or 2D data.
\begin{quote}\begin{description}
\sphinxlineitem{Parameters}\begin{itemize}
\item {} 
\sphinxAtStartPar
\sphinxstyleliteralstrong{\sphinxupquote{D}} (\sphinxstyleemphasis{ndarray}) \textendash{} 2D or pairwise data

\item {} 
\sphinxAtStartPar
\sphinxstyleliteralstrong{\sphinxupquote{order}} (\sphinxstyleemphasis{int}) \textendash{} order of the multi\sphinxhyphen{}way interactions

\item {} 
\sphinxAtStartPar
\sphinxstyleliteralstrong{\sphinxupquote{mtype}} (\sphinxstyleemphasis{str}) \textendash{} This specifies which multicorrelation measure to use. It defaults to
\sphinxcode{\sphinxupquote{Drezner}} {[}1{]}, but \sphinxcode{\sphinxupquote{Wang}} {[}2{]} and \sphinxcode{\sphinxupquote{Taylor}} {[}3{]} are options as well.

\item {} 
\sphinxAtStartPar
\sphinxstyleliteralstrong{\sphinxupquote{idxs}} (\sphinxstyleemphasis{ndarray}, optional) \textendash{} specify which indices of \sphinxcode{\sphinxupquote{D}} to compute multicorrelations of. The default is \sphinxcode{\sphinxupquote{None}}, in which case
all combinations of \sphinxcode{\sphinxupquote{order}} indices are computed.

\end{itemize}

\sphinxlineitem{Returns}
\sphinxAtStartPar
A vector of the multicorrelation scores computed and a vector of the column indices of
\sphinxcode{\sphinxupquote{D}} used to compute each multicorrelation.

\sphinxlineitem{Return type}
\sphinxAtStartPar
\sphinxstyleemphasis{(ndarray, ndarray)}

\end{description}\end{quote}
\subsubsection*{References}

\end{fulllineitems}

\index{uniformErdosRenyi() (in module HAT.HAT)@\spxentry{uniformErdosRenyi()}\spxextra{in module HAT.HAT}}

\begin{fulllineitems}
\phantomsection\label{\detokenize{HAT:HAT.HAT.uniformErdosRenyi}}
\pysigstartsignatures
\pysiglinewithargsret{\sphinxcode{\sphinxupquote{HAT.HAT.}}\sphinxbfcode{\sphinxupquote{uniformErdosRenyi}}}{\emph{\DUrole{n}{v}}, \emph{\DUrole{n}{e}}, \emph{\DUrole{n}{k}}}{}
\pysigstopsignatures
\sphinxAtStartPar
This function generates a uniform, random hypergraph.
\begin{quote}\begin{description}
\sphinxlineitem{Parameters}\begin{itemize}
\item {} 
\sphinxAtStartPar
\sphinxstyleliteralstrong{\sphinxupquote{v}} (\sphinxstyleemphasis{int}) \textendash{} number of vertices

\item {} 
\sphinxAtStartPar
\sphinxstyleliteralstrong{\sphinxupquote{e}} (\sphinxstyleemphasis{int}) \textendash{} number of edges

\item {} 
\sphinxAtStartPar
\sphinxstyleliteralstrong{\sphinxupquote{k}} (\sphinxstyleemphasis{int}) \textendash{} order of hypergraph

\end{itemize}

\sphinxlineitem{Returns}
\sphinxAtStartPar
Hypergraph

\sphinxlineitem{Return type}
\sphinxAtStartPar
\sphinxstyleemphasis{Hypergraph}

\end{description}\end{quote}

\end{fulllineitems}



\subsection{HAT.draw module}
\label{\detokenize{HAT:module-HAT.draw}}\label{\detokenize{HAT:hat-draw-module}}\index{module@\spxentry{module}!HAT.draw@\spxentry{HAT.draw}}\index{HAT.draw@\spxentry{HAT.draw}!module@\spxentry{module}}\index{incidencePlot() (in module HAT.draw)@\spxentry{incidencePlot()}\spxextra{in module HAT.draw}}

\begin{fulllineitems}
\phantomsection\label{\detokenize{HAT:HAT.draw.incidencePlot}}
\pysigstartsignatures
\pysiglinewithargsret{\sphinxcode{\sphinxupquote{HAT.draw.}}\sphinxbfcode{\sphinxupquote{incidencePlot}}}{\emph{\DUrole{n}{H}}, \emph{\DUrole{n}{shadeRows}\DUrole{o}{=}\DUrole{default_value}{True}}, \emph{\DUrole{n}{connectNodes}\DUrole{o}{=}\DUrole{default_value}{True}}, \emph{\DUrole{n}{dpi}\DUrole{o}{=}\DUrole{default_value}{200}}, \emph{\DUrole{n}{edgeColors}\DUrole{o}{=}\DUrole{default_value}{None}}}{}
\pysigstopsignatures
\sphinxAtStartPar
Plot the incidence matrix of a hypergraph.
\begin{quote}\begin{description}
\sphinxlineitem{Parameters}\begin{itemize}
\item {} 
\sphinxAtStartPar
\sphinxstyleliteralstrong{\sphinxupquote{H}} \textendash{} a HAT.hypergraph object

\item {} 
\sphinxAtStartPar
\sphinxstyleliteralstrong{\sphinxupquote{shadeRows}} \textendash{} shade rows (bool)

\item {} 
\sphinxAtStartPar
\sphinxstyleliteralstrong{\sphinxupquote{connectNodes}} \textendash{} connect nodes in each hyperedge (bool)

\item {} 
\sphinxAtStartPar
\sphinxstyleliteralstrong{\sphinxupquote{dpi}} \textendash{} the resolution of the image (int)

\item {} 
\sphinxAtStartPar
\sphinxstyleliteralstrong{\sphinxupquote{edgeColors}} \textendash{} The colors of edges represented in the incidence matrix. This is random by default

\end{itemize}

\sphinxlineitem{Returns}
\sphinxAtStartPar
matplotlib axes with figure drawn on to it

\end{description}\end{quote}

\end{fulllineitems}



\subsection{HAT.multilinalg module}
\label{\detokenize{HAT:module-HAT.multilinalg}}\label{\detokenize{HAT:hat-multilinalg-module}}\index{module@\spxentry{module}!HAT.multilinalg@\spxentry{HAT.multilinalg}}\index{HAT.multilinalg@\spxentry{HAT.multilinalg}!module@\spxentry{module}}\index{hosvd() (in module HAT.multilinalg)@\spxentry{hosvd()}\spxextra{in module HAT.multilinalg}}

\begin{fulllineitems}
\phantomsection\label{\detokenize{HAT:HAT.multilinalg.hosvd}}
\pysigstartsignatures
\pysiglinewithargsret{\sphinxcode{\sphinxupquote{HAT.multilinalg.}}\sphinxbfcode{\sphinxupquote{hosvd}}}{\emph{\DUrole{n}{T}}, \emph{\DUrole{n}{M}\DUrole{o}{=}\DUrole{default_value}{True}}, \emph{\DUrole{n}{uniform}\DUrole{o}{=}\DUrole{default_value}{False}}, \emph{\DUrole{n}{sym}\DUrole{o}{=}\DUrole{default_value}{False}}}{}
\pysigstopsignatures
\sphinxAtStartPar
Higher Order Singular Value Decomposition
\begin{quote}\begin{description}
\sphinxlineitem{Parameters}\begin{itemize}
\item {} 
\sphinxAtStartPar
\sphinxstyleliteralstrong{\sphinxupquote{uniform}} \textendash{} Indicates if T is a uniform tensor

\item {} 
\sphinxAtStartPar
\sphinxstyleliteralstrong{\sphinxupquote{sym}} \textendash{} Indicates if T is a super symmetric tensor

\item {} 
\sphinxAtStartPar
\sphinxstyleliteralstrong{\sphinxupquote{M}} \textendash{} Indicates if the factor matrices are required as well as the core tensor

\end{itemize}

\sphinxlineitem{Returns}
\sphinxAtStartPar
The singular values of the core diagonal tensor and the factor matrices.

\end{description}\end{quote}

\end{fulllineitems}

\index{supersymHosvd() (in module HAT.multilinalg)@\spxentry{supersymHosvd()}\spxextra{in module HAT.multilinalg}}

\begin{fulllineitems}
\phantomsection\label{\detokenize{HAT:HAT.multilinalg.supersymHosvd}}
\pysigstartsignatures
\pysiglinewithargsret{\sphinxcode{\sphinxupquote{HAT.multilinalg.}}\sphinxbfcode{\sphinxupquote{supersymHosvd}}}{\emph{\DUrole{n}{T}}}{}
\pysigstopsignatures
\sphinxAtStartPar
Computes the singular values of a uniform, symetric tensor. See Algorithm 1 in {[}1{]}.
\begin{quote}\begin{description}
\sphinxlineitem{Parameters}
\sphinxAtStartPar
\sphinxstyleliteralstrong{\sphinxupquote{T}} \textendash{} A uniform, symmetric multidimensional array

\sphinxlineitem{Returns}
\sphinxAtStartPar
The singular values that compose the core tensor of the HOSVD on T.

\end{description}\end{quote}
\subsubsection*{References}

\end{fulllineitems}

\index{HammingSimilarity() (in module HAT.multilinalg)@\spxentry{HammingSimilarity()}\spxextra{in module HAT.multilinalg}}

\begin{fulllineitems}
\phantomsection\label{\detokenize{HAT:HAT.multilinalg.HammingSimilarity}}
\pysigstartsignatures
\pysiglinewithargsret{\sphinxcode{\sphinxupquote{HAT.multilinalg.}}\sphinxbfcode{\sphinxupquote{HammingSimilarity}}}{\emph{\DUrole{n}{A1}}, \emph{\DUrole{n}{A2}}}{}
\pysigstopsignatures
\sphinxAtStartPar
Computes the Spectral\sphinxhyphen{}S similarity of 2 Adjacency tensors {[}1{]}.
\begin{quote}\begin{description}
\sphinxlineitem{Parameters}\begin{itemize}
\item {} 
\sphinxAtStartPar
\sphinxstyleliteralstrong{\sphinxupquote{A1}} (\sphinxstyleemphasis{ndarray}) \textendash{} adjacency tensor 1

\item {} 
\sphinxAtStartPar
\sphinxstyleliteralstrong{\sphinxupquote{A2}} (\sphinxstyleemphasis{ndarray}) \textendash{} adjacency tensor 2

\end{itemize}

\sphinxlineitem{Returns}
\sphinxAtStartPar
Hamming similarity measure

\sphinxlineitem{Return type}
\sphinxAtStartPar
\sphinxstyleemphasis{float}

\end{description}\end{quote}
\subsubsection*{References}

\end{fulllineitems}

\index{SpectralHSimilarity() (in module HAT.multilinalg)@\spxentry{SpectralHSimilarity()}\spxextra{in module HAT.multilinalg}}

\begin{fulllineitems}
\phantomsection\label{\detokenize{HAT:HAT.multilinalg.SpectralHSimilarity}}
\pysigstartsignatures
\pysiglinewithargsret{\sphinxcode{\sphinxupquote{HAT.multilinalg.}}\sphinxbfcode{\sphinxupquote{SpectralHSimilarity}}}{\emph{\DUrole{n}{L1}}, \emph{\DUrole{n}{L2}}}{}
\pysigstopsignatures
\sphinxAtStartPar
Computes the Spectral\sphinxhyphen{}S similarity of 2 Laplacian tensors {[}1{]}.
\begin{quote}\begin{description}
\sphinxlineitem{Parameters}\begin{itemize}
\item {} 
\sphinxAtStartPar
\sphinxstyleliteralstrong{\sphinxupquote{L1}} (\sphinxstyleemphasis{ndarray}) \textendash{} Laplacian tensor 1

\item {} 
\sphinxAtStartPar
\sphinxstyleliteralstrong{\sphinxupquote{L2}} (\sphinxstyleemphasis{ndarray}) \textendash{} Laplacian tensor 2

\end{itemize}

\sphinxlineitem{Returns}
\sphinxAtStartPar
Spectral\sphinxhyphen{}S similarity measure

\sphinxlineitem{Return type}
\sphinxAtStartPar
\sphinxstyleemphasis{float}

\end{description}\end{quote}
\subsubsection*{References}

\end{fulllineitems}

\index{kronExponentiation() (in module HAT.multilinalg)@\spxentry{kronExponentiation()}\spxextra{in module HAT.multilinalg}}

\begin{fulllineitems}
\phantomsection\label{\detokenize{HAT:HAT.multilinalg.kronExponentiation}}
\pysigstartsignatures
\pysiglinewithargsret{\sphinxcode{\sphinxupquote{HAT.multilinalg.}}\sphinxbfcode{\sphinxupquote{kronExponentiation}}}{\emph{\DUrole{n}{M}}, \emph{\DUrole{n}{x}}}{}
\pysigstopsignatures
\sphinxAtStartPar
Kronecker Product Exponential.
\begin{quote}\begin{description}
\sphinxlineitem{Parameters}\begin{itemize}
\item {} 
\sphinxAtStartPar
\sphinxstyleliteralstrong{\sphinxupquote{M}} (\sphinxstyleemphasis{ndarray}) \textendash{} a matrix

\item {} 
\sphinxAtStartPar
\sphinxstyleliteralstrong{\sphinxupquote{x}} (\sphinxstyleemphasis{int}) \textendash{} power of exponentiation

\end{itemize}

\sphinxlineitem{Returns}
\sphinxAtStartPar
Krnoecker Product exponentiation of \sphinxstylestrong{M} a total of \sphinxstylestrong{x} times

\sphinxlineitem{Return type}
\sphinxAtStartPar
\sphinxstyleemphasis{ndarray}

\end{description}\end{quote}

\sphinxAtStartPar
This function performs the Kronecker Product on a matrix \(M\) a total of
\(x\) times. The Kronecker product is defined for two matrices
\(A\in\mathbf{R}^{l \times m}, B\in\mathbf{R}^{m \times n}\) as the matrix
\begin{equation*}
\begin{split}A \bigotimes B= \begin{pmatrix} A_{1,1}B & A_{1,2}B & \dots & A_{1,m}B \\ A_{2,1}B & A_{2,2}B & \dots & A_{2,m}B \\ \vdots & \vdots & \ddots & \vdots \\ A_{l,1}B & A_{l,2}B & \dots & A_{l,n}B \end{pmatrix}\end{split}
\end{equation*}
\end{fulllineitems}


\sphinxstepscope


\section{Hypergraph References}
\label{\detokenize{ref:hypergraph-references}}\label{\detokenize{ref::doc}}\begin{quote}
\end{quote}


\chapter{Installation}
\label{\detokenize{index:installation}}

\section{MATLAB}
\label{\detokenize{index:matlab}}
\sphinxAtStartPar
The MATLAB distribution of HAT can be installed through either the \sphinxhref{https://www.mathworks.com/matlabcentral/fileexchange/121013-hypergraph-analysis-toolbox}{MATLAB Central}. A MathWorks \sphinxcode{\sphinxupquote{.mltbx}} file can be downloaded from the site,
and installed through the add on manager in the MATLAB Home environment. Once installed as a toolbox, you will have access to all HAT functionality.

\sphinxAtStartPar
The MATLAB distribution has the following dependencies:
\begin{enumerate}
\sphinxsetlistlabels{\arabic}{enumi}{enumii}{}{.}%
\item {} 
\sphinxAtStartPar
\sphinxhref{https://users.math.msu.edu/users/chenlipi/teneig.html}{TenEig — Tensor Eigenpairs Solver}

\end{enumerate}


\section{Python}
\label{\detokenize{index:python}}
\sphinxAtStartPar
The Python distribution of HAT may be installed through pip:

\begin{sphinxVerbatim}[commandchars=\\\{\}]
\PYG{o}{\PYGZgt{}\PYGZgt{}} \PYG{n}{pip} \PYG{n}{install} \PYG{n}{HypergraphAnalysisToolbox}
\end{sphinxVerbatim}

\sphinxAtStartPar
Information on the PiPy distribution is available \sphinxhref{https://pypi.org/project/HypergraphAnalysisToolbox/}{here}. Once installed, HAT may be imported into the Python invironment with the command:

\begin{sphinxVerbatim}[commandchars=\\\{\}]
\PYG{k+kn}{import} \PYG{n+nn}{HAT}              \PYG{c+c1}{\PYGZsh{} Import package}
\PYG{k+kn}{import} \PYG{n+nn}{HAT}\PYG{n+nn}{.}\PYG{n+nn}{Hypergraph}   \PYG{c+c1}{\PYGZsh{} Hypergraph class}
\PYG{k+kn}{import} \PYG{n+nn}{HAT}\PYG{n+nn}{.}\PYG{n+nn}{plot}         \PYG{c+c1}{\PYGZsh{} Visualization tools}
\end{sphinxVerbatim}

\sphinxAtStartPar
The Python distribution has the following dependencies:
\begin{enumerate}
\sphinxsetlistlabels{\arabic}{enumi}{enumii}{}{.}%
\item {} 
\sphinxAtStartPar
numpy

\item {} 
\sphinxAtStartPar
scipy

\item {} 
\sphinxAtStartPar
matplotlib

\item {} 
\sphinxAtStartPar
itertools

\end{enumerate}


\section{Development Distribution}
\label{\detokenize{index:development-distribution}}
\sphinxAtStartPar
All implementations of HAT are managed through a \sphinxhref{https://github.com/Jpickard1/Hypergraph-Analysis-Toolbox}{common git repository}. This is public, so it may be
cloned and modified. If interested in modifying or contributing to HAT, please contact Joshua Pickard at \sphinxhref{mailto:jpic@umich.edu}{jpic@umich.edu}


\chapter{Indices and tables}
\label{\detokenize{index:indices-and-tables}}\begin{itemize}
\item {} 
\sphinxAtStartPar
\DUrole{xref,std,std-ref}{genindex}

\item {} 
\sphinxAtStartPar
\DUrole{xref,std,std-ref}{modindex}

\item {} 
\sphinxAtStartPar
\DUrole{xref,std,std-ref}{search}

\end{itemize}


\renewcommand{\indexname}{Python Module Index}
\begin{sphinxtheindex}
\let\bigletter\sphinxstyleindexlettergroup
\bigletter{h}
\item\relax\sphinxstyleindexentry{HAT.draw}\sphinxstyleindexpageref{HAT:\detokenize{module-HAT.draw}}
\item\relax\sphinxstyleindexentry{HAT.HAT}\sphinxstyleindexpageref{HAT:\detokenize{module-HAT.HAT}}
\item\relax\sphinxstyleindexentry{HAT.Hypergraph}\sphinxstyleindexpageref{HAT:\detokenize{module-HAT.Hypergraph}}
\item\relax\sphinxstyleindexentry{HAT.multilinalg}\sphinxstyleindexpageref{HAT:\detokenize{module-HAT.multilinalg}}
\end{sphinxtheindex}

\renewcommand{\indexname}{Index}
\printindex
\end{document}